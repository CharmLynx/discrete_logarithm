\documentclass[fontsize=14.4pt]{report}
\usepackage[T2A]{fontenc}
\usepackage[utf8]{inputenc}
\usepackage[english, ukrainian]{babel}

\usepackage{fontsize}
\usepackage{geometry}
\usepackage{listings}
\usepackage{graphicx}
\usepackage{pgfplots}
\usepackage{caption}
\usepackage{subcaption}
\usepackage{hyperref}
\usepackage{tabularx}
\usepackage{float}
\usepackage{amsmath}
\usepackage{amsmath, amssymb}
\usepackage{pgfplots}
\pgfplotsset{compat=1.18}


\hypersetup{colorlinks=true, urlcolor=cyan}

\geometry{top=40pt}
\geometry{bottom=20pt}
\geometry{right=50pt}
\geometry{left=50pt}

% chktex 8 off
\begin{document}
\thispagestyle{empty}
\begin{center}
Національний технічний університет України \par
"Київський політехнічний інститут ім. Ігоря Сікорського"\par
\vspace{5mm}
Навчально-науковий Фізико-технічний інститут\par
Кафедра математичних методiв захисту iнформацiї

\vspace{60mm}
\large\MakeUppercase{\textbf{Комп’ютерний практикум 2: реалiзацiя алгоритму дискретного логарифмування Сiльвера-Полiга-Геллмана}}\par
\vspace{1mm}

\end{center}

\vspace{50mm}
\begin{flushright}
Виконали студенти\par
групи ФІ-23\par
Чуй Тимофій і Малютіна Марина
\end{flushright}

\vspace{60mm}
\begin{center}
{Київ~--- 2025}
\end{center}

\newpage
\pagenumbering{gobble}

\noindent\large\textbf{Мета роботи}\par
Ознайомлення з алгоритмом дискретного логарифмування Сiльвера-Полiга-Геллмана. Практична
реалiзацiя цього алгоритму. Пошук переваг, недолiкiв та особливостей застосування даного алгоритму
дискретного логарифмування. Практична оцiнка складностi роботи алгоритму.\newline
\noindent\large\textbf{Постановка задачі}\par
\begin{enumerate}
    \item Уважно прочитати методичні вказівки до виконання \newline комп’ютерного практикуму.
    \item Написати програму, що розв’язує задачу дисктерного логарифму шляхом звичайного перебору.
    \item Написати програму, що реалiзовує алгоритм Сiльвера-Полiга-Геллмана для груп типу $ \mathbb{Z}_p^* $
    \item Застосовувати реалiзований алгоритм Сiльвера-Полiга-Геллмана та метод перебору до задач дискретного логарифма, якi формує допоміжна програма, почергово зi збiльшенням порядку p. У випадку, якщо допомiжна програма не справляється зi завданням генерацiї задачi, або
    реалiзацiя студента не справляється з розв’язком задачi за вiдведений час, зупинити збiльшення
    вхiдного параметра.
    \item Замiряти час роботи реалiзацiї алгоритму Сiльвера-Полiга-Геллмана та метод перебору для обох
    типiв задач створених допомiжною програмою..
    \item Поpiвняти результати часу роботи мiж методами i мiж типами задач. Створити вiзуалiзацiю
    залежностi часу роботи вiд вхiдного параметра. Пояснити результати.
\end{enumerate}
\newpage

\noindent\large\textbf{Опис рішень}\par
З методом перебору все очікувано просто, він працює по принципу "береш та порівнюєш". Щодо алгоритму Сільвера-Поліга-Геллмана, довелось трохи поламати голову над тим як простіше реалізувати, який тип даних використати для вхідних. В іншому - без суттєвих проблем.


\noindent\large\textbf{Результати}\par
Нижче наведена таблиця з результатами заміру середнього часу 5 запусків кожного методу а також візуалізація, аби наглядно можна було побачити перевагу СПГ над перебором.
\begin{table}[h!]
\small
\centering
\begin{tabular}{|c|c|c|c|c|c|c|}
\hline
К-ть знаків & $\alpha$ & $\beta$ & $n$ & $x$ & Час перебором &  Час СПГ \\
\hline
3/тип 1 & 61 & 27 & 601 & 12 & 0.00005 & 0.0001 \\
\hline
3/тип 2 & 126 & 561 & 719 & 200 & 0.0002 & 0.00004 \\
\hline
4/тип 1 & 722 & 1789 & 2179 & 1206 & 0.0012 & 0.0001 \\
\hline
4/тип 2 & 3048 & 932 & 5903 & 3841 & 0.0018 & 0.0003 \\
\hline
5/тип 1 & 39105 & 46297 & 60289 & 38294 & 0.0158 & 0.0014 \\
\hline
5/тип 2 & 26664 & 9606 & 31973 & 1266 & 0.0011 & 0.0001 \\
\hline
6/тип 1 & 207764 & 25430 & 390869 & 40772 & 0.0203 & 0.0011 \\
\hline
6/тип 2 & 69036 & 123995 & 521533 & 31270 & 0.0128 & 0.0012 \\
\hline
7/тип 1 & 3676331 & 614793 & 4170541 & 3642800 & 2.0489 & 0.0581 \\
\hline
7/тип 2 & 4131913 & 1763212 & 4877459 & 817961 & 0.4212 & 0.0138 \\
\hline
8/тип 1 & 14390843 & 42225218 & 47487259 & 13798106 & 8.4382 & 0.2172 \\
\hline
8/тип 2 & 32983050 & 56317687 & 83555669 & 16639597 & 10.2178 & 0.2650 \\
\hline
9/тип 1 & 121563032 & 516170742 & 893533441 & 24637684 & 15.4651 & 0.3924 \\
\hline
9/тип 2 & 474820254 & 40136582 & 937237349 & 528673215 & 300 & 8.3025 \\
\hline
10/тип 1 & 933895063 & 1484143709 & 1639867079 & 603807270 & 300 & 9.4640 \\
\hline
10/тип 2 & 3068468170 & 3940222237 & 7207030489 & 615948263 & 300 & 11.8292 \\
\hline
11/тип 1 & 26708645486 & 19564897418 & 27574761517 & 1923177060 & 300 & 66.4284 \\
\hline
11/тип 2 & 17435029047 & 10429220418 & 31788610771 & 14754109763 & 300 & 45.3546 \\
\hline
\end{tabular}
\label{tab:results}
\end{table}


\begin{tikzpicture}
  \begin{axis}[
    title = {Задача типу один},
    xlabel = кількість знаків,
    ylabel = час,
    grid = major,
    xtick={3, 4, 5, 6, 7, 8, 9, 10, 11},
    ytick={0,50, 100, 150, 200, 250, 300},
    legend style={at={(1.05,1)}, anchor=north west}
  ]

    % Перша лінія
    \addplot[
      mark=*,
      color=blue
    ] table {
      x y
      3 0.00005 
      4 0.0012
      5 0.0158
      6 0.0203
      7 2.0489
      8 8.4382
      9 15.4651
      10 300
      11 300
    };
    \addlegendentry{перебір}

    % Друга лінія
    \addplot[
      mark=square*,
      color=red
    ] table {
      x y
      3 0.0001 
      4 0.0001
      5 0.0014
      6 0.0011
      7 0.0581
      8 0.2172
      9 0.3924
      10 9.4640
      11 66.4284
    };
    \addlegendentry{СПГ}

  \end{axis}
\end{tikzpicture}

\begin{tikzpicture}
  \begin{axis}[
    title = {Задача типу два},
    xlabel = кількість знаків,
    ylabel = час,
    grid = major,
    xtick={3, 4, 5, 6, 7, 8, 9, 10, 11},
    ytick={0,50, 100, 150, 200, 250, 300},
    legend style={at={(1.05,1)}, anchor=north west}
  ]

    % Перша лінія
    \addplot[
      mark=*,
      color=blue
    ] table {
      x y
      3 0.0002
      4 0.0018
      5 0.0011
      6 0.0128
      7 0.4212
      8 10.2178
      9 300
      10 300
      11 300

    };
    \addlegendentry{перебір}

    % Друга лінія
    \addplot[
      mark=square*,
      color=red
    ] table {
      x y
      3 0.00004
      4 0.0003
      5 0.0001
      6 0.0012
      7 0.0138
      8 0.265
      9 8.3025
      10 9.4640
      11 45.3546
    };
    \addlegendentry{СПГ}

  \end{axis}
\end{tikzpicture}

\begin{tikzpicture}
  \begin{axis}[
    title = {Перебір},
    xlabel = кількість знаків,
    ylabel = час,
    grid = major,
    xtick={3, 4, 5, 6, 7, 8, 9, 10, 11},
    ytick={0,50, 100, 150, 200, 250, 300},
    legend style={at={(1.05,1)}, anchor=north west}
  ]

    % Перша лінія
    \addplot[
      mark=*,
      color=blue
    ] table {
      x y
      3 0.00005 
      4 0.0012
      5 0.0158
      6 0.0203
      7 2.0489
      8 8.4382
      9 15.4651
      10 300
      11 300

    };
    \addlegendentry{тип 1}

    % Друга лінія
    \addplot[
      mark=square*,
      color=red
    ] table {
      x y
      3 0.0002
      4 0.0018
      5 0.0011
      6 0.0128
      7 0.4212
      8 10.2178
      9 300
      10 300
      11 300
    };
    \addlegendentry{тип 2}

  \end{axis}
\end{tikzpicture}

\begin{tikzpicture}
  \begin{axis}[
    title = {СПГ},
    xlabel = кількість знаків,
    ylabel = час,
    grid = major,
    xtick={3, 4, 5, 6, 7, 8, 9, 10, 11},
    ytick={0, 15, 30, 45, 60},
    legend style={at={(1.05,1)}, anchor=north west}
  ]

    % Перша лінія
    \addplot[
      mark=*,
      color=blue
    ] table {
      x y
      3 0.0001 
      4 0.0001
      5 0.0014
      6 0.0011
      7 0.0581
      8 0.2172
      9 0.3924
      10 9.4640
      11 66.4284
    };
    \addlegendentry{тип 1}

    % Друга лінія
    \addplot[
      mark=square*,
      color=red
    ] table {
      x y
      3 0.00004
      4 0.0003
      5 0.0001
      6 0.0012
      7 0.0138
      8 0.265
      9 8.3025
      10 9.4640
      11 45.3546
    };
    \addlegendentry{тип 2}

  \end{axis}
\end{tikzpicture}

\noindent\large\textbf{Висновки}\par
Загальна картина така, що СПГ працює краще за перебір, були числа на яких перебір спрацював швидше, але це не заслуга методу, а просто удача, що розв'язок недалеко від 0.
Розглядалась кількість знаків від 3 до 11, вбільших значеннях допоміжна програма не генерувала числа і наша реалізація виходила за рамки 5 хвилин. Із отриманих значень можна побачити як початкових значеннях перебір йшов майже в ногу з СПГ, а зі збільшенням довжини числа перебір першим вийшов за рамки 5 хвилин обчислень
\end{document}
